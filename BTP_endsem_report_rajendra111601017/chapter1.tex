\chapter{Introduction}
\pagenumbering{arabic}\hspace{3mm}

During my summer internship at UST Global (2019), I studied various SLAM and swarm algorithm. Now my attempt is to use reinforcement learning based algorithm for robot control, path planning, mapping and localization.

\section{Problem Statement}
Study Simultaneous localization and mapping(SLAM) and its various type and application. Study quadcopter control and path planning. Improve same using reinforcement learning based algorithms. Study multi robot system and their various coordination algorithms(swarm algorithms).

\section{Goal for this semester}

For this semester my main focus is to study SLAM and Swarm behaviour of robots. I'll writing the efficient reinforcement learning based algorithms. Test and analysis these algorithms with simulated robots using ros. I will be learning to design the custom environment for simulations. By the end this semester I'll start implementing these algorithms on the real hardware.

\section{Organization of The Report}

Code written for slam package is large and is in form of ros packages, hence this report contain pseudo codes and their documentation and algorithms. Full detailed for each of this can be found git.\cite{btpgit}
\\
\textbf{chapter 1} : We introduced the problem statement, discussed goal for this semester and organisation of this report.
\\
\textbf{chapter 2} : We'll introduce SLAM problem and discuss its common terminology, algorithms and my prior work in it using gmapping and RTABMAP slam. 
\\
\textbf{chapter 3} : We'll discuss about the various Reinforcement learning algorithm and it implementation on mountain car problem using the openAI gym.
\\
\textbf{chapter 4} : We'll discuss about control and planning in drones firstly using pid controller and later using the reinforcement learning.
\\
\textbf{chapter 5} : We'll discuss about the sub-field of AI which swarm intelligence and its various application in robotics.
\\
\textbf{chapter 6} : In the end, we'll conclude my work and discuss future work.


